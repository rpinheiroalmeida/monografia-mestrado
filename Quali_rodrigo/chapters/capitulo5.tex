%\textual
\chapter{Metodologia e Cronograma}

\section{Metodologia}

\subsection{Primeira fase: estudo e análise}

Inicialmente, será realizado um estudo sobre proveniência de dados,
computação na nuvem e bancos \textit{NoSQL}, mais especificamente bancos de grafos. O objetivo dessa fase é identificar os principais desafios na captura automática de proveniência de dados na computação na nuvem.

\subsection{Segunda fase: especificação da arquitetura}

Na segunda fase deve ser especificado um modelo de arquitetura que atenda
os desafios definidos na primeira fase. Essa fase envolve a
especificação e uma política de proveniência de dados na nuvem no contexto da Bioinformática usando o modelo de proveniência PROV-DM.


\subsection{Terceira fase: implementação}

Na terceira fase serão definidas as características técnicas para o
desenvolvimento da arquitetura, ou seja, as ferramentas necessárias para a construção da arquitetura proposta na segunda fase.

\subsection{Quarta fase: testes}

Nesta fase pretende-se definir os casos de uso a serem executados e
executá-los em diversas nuvens computacionais com o objetivo de
validar e realizar ajustes necessários em relação a arquitetura
proposta na terceira fase.

\subsection{Quinta fase: avaliação e correção}

Na fase cinco pretende-se avaliar o desempenho da arquitetura proposta
nos casos de uso definidos na quarta fase, assim como fazer as devidas
correções e reavaliações da proposta para que os objetivos sejam cumpridos. 


\subsection{Sexta Fase: publicação e defesa da dissertação}

Nesta fase pretende-se discutir os resultados conseguidos, apresentar artigos acadêmicos e elaborar a dissertação do mestrado. O objetivo desta fase é a apresentação do trabalho à comunidade através de meios formais e a defesa da dissertação. 

\section{Cronograma de Execução}

Durante esse primeiro ano do mestrado foram completados os 24 créditos de
disciplinas necessários para a obtenção do título de mestre de acordo
com o regimento do Programa de Pós-graduação em Informática da UnB,
faltando apenas a defesa da dissertação do mestrado. 

Nesse período foi realizada a publicação do artigo \cite{rodrigo} que
propõe a captura automática de dados de proveniência e o armazenamento
em um esquema relacional baseado no modelo PROV-DM. Para a validação da proposta, o simulador implementado em \cite{renato} foi modificado para realizar a captura automática dos dados de proveniência e salvá-los em um banco de dados relacional.

Neste período também foi realizado um levantamento bibliográfico, a definição
dos objetivos da pesquisa e a definição da arquitetura proposta. As etapas desse projeto são apresentadas na Tabela \ref{tab:cronograma}, a saber:

 	\begin{enumerate}
    \item Pesquisa bibliográfica para conhecer a teoria relacionada com o projeto, detalhamento do projeto de pesquisa;
    \item Definição das ferramentas de implementação  e arquitetura que será executada na nuvem; 
    \item Implementação da arquitetura definida;
    \item Testes;
    \item Discussão, avaliação e correção;
    \item Elaboração da dissertação, escrita de artigos ciêntíficos e defesa da dissertação.

    \end{enumerate}

\begin{table}[!h]
  \centering
  \large
  \setlength{\arrayrulewidth}{2\arrayrulewidth}
  \setlength{\belowcaptionskip}{10pt}
  \caption{\ Cronograma de atividades.}
  \scalebox{0.9}{
  \begin{tabular}{|c|c|c|c|c|c|c|}
     \hline
%	\multicolumn{1}{|c|}{""} & \multicolumn{6}{|c|}{\textbf{Ano 2013}}\\ 
     \hline
     \textbf{Etapa} & \textbf{2013/1} & \textbf{2013/2} & \textbf{2014/1} & \textbf{2014/2} \\
     \hline
     1 & X & X & X & ""\\    
     \hline
     2 & "" & X & X & ""\\    
     \hline
     3 & "" & "" & X & "" \\
     \hline
     4 & "" & "" & X & X \\
     \hline
     5 & "" & "" &  & X\\
     \hline
     6 & "" & "" & "" & X\\
     \hline
  \end{tabular}}
\label{tab:cronograma}
\end{table}
